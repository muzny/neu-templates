
\documentclass{neu_syllabus}

% Professor/Course information
\prof{Nate}{Derbinsky}
\date{Fall 2018}
\course{College of Computer and Information Science}{CS1000}{Computers!}

\begin{document}

% Create info box
\begin{SyllabusHeader}
\instructor
\office{WVH 208B}{Monday, Thursday, Friday @ 10:30am -- 11:30am and by appointment}
\contact{(617) 373-7382}{n.derbinsky@northeastern.edu}{http://derbinsky.info}
\topic{Credits}{4}
\end{SyllabusHeader}

% Description
\SyllabusSection{Course Description}
Insert course description here.
Should match the course description that appears in the Catalog.

% Prerequisites
\SyllabusSection{Course Prerequisites/Corequisites}
Insert any course pre/corequisites here.
This should match the course description that appears in the Catalog.
If there are no prerequisites or corequisites please note this.

% Required books
\begin{SyllabusBooks}{Required Textbook(s)}
\bookInfo{Savitch, Walter}{Java: Introduction to Problem Solving and Programming}{7th}{Addison-Wesley}{2014}{978-0133766264}
\end{SyllabusBooks}

% Recommended books
\begin{SyllabusBooks}{Recommended Learning Materials}
\book{\textit{Insert other helpful resources should be listed here.}}
\end{SyllabusBooks}

\newpage

\SyllabusSection{Campus Resources}

Students are encouraged to take advantage of the Digital Scholarship Group at Northeastern\footnote{\url{https://dsg.neu.edu}}.
They offer a wealth of services -- including digital data collections -- and can offer advice on collecting and structuring digital data.
They also offer a quiet space to work.
\\

Assignments that involve writing and presentation will be judged on clarity of presentation as well as content.
Students who are having difficulty with writing will be referred to the Northeastern University Writing Center\footnote{\url{https://www.northeastern.edu/writingcenter/}}.

% Learning outcomes
\SyllabusSection{Course Learning Outcomes}

At the completion of this course, the student should be able to:
\begin{itemize}
\item Foo
\item Bar
\item Baz
\item Quz
\end{itemize}

% Instructional methodologies
\SyllabusSection{Instructional Methodologies}
This course will combine traditional lecturing with hands-on assignments that reinforce the lecture material.
In particular, lectures will focus on concepts and ideas while the assignments will provide concrete experience and skills.

% Attendance
\SyllabusSection{Attendance Policy}
Students are expected to attend classes regularly, take tests, and submit papers and other work at the times specified by the instructor.
Students who are absent repeatedly from class or studio will be evaluated by faculty responsible for the course to ascertain their ability to achieve the course objectives and to continue in the course.
Instructors may include, as part of the semester's grades, marks for the quality and quantity of the student's participation in class.

% Grading
\SyllabusSection{Grading Policy}
Insert grading policy here.  Your policy must state:
\begin{itemize}
\item specific assignments a student must complete to meet the learning outcomes.
\item number of assignments in each category that are required.
\item{%
relative weight of each assignment \\
\begin{tabular}{ p{1.0cm} l }
	25\% & Things \\
	75\% & Stuffs \\
\end{tabular}
}
\item if the assignment is a project, presentation, paper, etc., criteria must be established so that students will understand exactly how they will be graded (may be handed out to students under separate cover).
\end{itemize}

% Makeup
\SyllabusSection{Make-Up Policy}

All assignments have a specific due date and time.
Submissions will be accepted \textit{up to one day} after the deadline with a 50\% penalty.
The assignment will be graded and returned as normal, but the grade will be recorded as half of what was earned.
For example, an on-time submission might receive a grade of 90 points.
The same assignment submitted after the deadline would receive 45 points ($90 \times 0.5$).
\\

Students who miss scheduled exams will not, as a matter of course, be able to make up those exams.
If there is a legitimate reason why a student will not be able to complete an assignment on time or not be present for an exam, then they should contact the instructor beforehand.
Under extreme circumstances, as decided on a case-by-case basis by the instructor, students may be allowed to make up assignments or exams without first informing the instructor.

\SyllabusSection{Academic Conduct}
This class has very strict standards for borrowing code: if you borrow anything for use in your project, you must have a citation.
A good guideline is that if you take more than three lines of code from some source, you must include the information on where it came from.
A URL or a notation (e.g., ``MATLAB help files'') is fine.
If it is an entire function, note it at the beginning of the code segment and include any original credit information.
Provide a qualitative description of what you used, and what you changed/contributed.
If you have a question about what is considered a violation of this policy, \textbf{ASK}!
\\

Unless stated otherwise (e.g., group project), assignments reflect individual work.
While you may discuss concepts and ideas with other students, there is to be no direct collaboration.
If you steal someone else's work, you fail the class.
If someone uses your work, you fail the class.
If you are unsure about this policy, \textbf{ask the instructor}.
The university's academic integrity policy discusses actions regarded as violations and consequences for students\footnote{\url{http://www.northeastern.edu/osccr/academic-integrity}}.

\newpage

\SyllabusClassroomEnvironment

\newpage

% Schedule
\SyllabusSection{Weekly Schedule}
The following schedule is tentative and subject to change (including topics, assignments, and exams).
It will benefit you greatly to complete the assigned reading \textit{before} attending the lecture.
\\

\begin{SyllabusSchedule}
\week{Introduction to Computation and Programming}{1.1, 1.2, 1.3}{}
\week{Variables, I/O, Types, Strings}{2.1, 2.2, 2.3, 2.4}{HW0 due}
\week{Control Flow, Conditionals}{3.1, 3.2, 3.3}{HW1 due}
\week{Exam 1 Review, Expressions, Loops}{4.1, 4.2}{Exam 1, HW2 due}
\week{Loops cont'd}{4.1, 4.2}{HW3 due}
\week{Functions}{5.1, 6.4}{HW4 due}
\week{Exam 2 Review, Arrays}{7.1, 7.2, 7.3}{Exam 2, HW5 due}
\week{Arrays cont'd}{7.1, 7.2, 7.3}{HW6 due}
\week{Testing and Debugging}{1.3}{HW7 due}
\week{Exam 3 Review, Object Oriented Programming}{5.1, 5.2, 5.3, 6.1, 6.7}{Exam 3}
\week{Designing Classes}{5.1, 5.2, 5.3, 6.1, 6.7}{HW8 due}
\week{Exceptions, File I/O}{9.1, 10.1, 10.2}{HW9 due}
\week{Lists}{12.1}{HW10 due}
\week{Advanced Topics (e.g. GUI Programming)}{13}{HW11 due}
\week{Final Exam Review}{}{Final Exam}
\end{SyllabusSchedule}
\\

All students are strongly encouraged to use the TRACE (Teacher Rating and Course Evaluation) system\footnote{\url{https://www.northeastern.edu/trace/}} near the end of the course to evaluate this course.
A reminder about TRACE should arrive via email about two weeks before the end of the course.


\end{document}
